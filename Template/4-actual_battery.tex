\section{Actual Battery}
    \subsection{Assumptions}
        An actual battery is much more complex than the idealized model presented in the previous section. In this section, we will consider more factors
        that affect the performance and behavior of real batteries:
        \begin{itemize}
            \item \textbf{Temperature}: The performance of a battery can vary significantly with temperature. At low temperatures, the internal resistance
                                        increases, leading to reduced capacity and power output. Conversely, high temperatures can enhance performance but
                                        may also accelerate degradation.
            \item \textbf{Complex Power Profile}: Real batteries often experience varying power demands, which can affect their efficiency and lifespan.
                                                  High performance demands can lead to increased heat generation, which int turns affect battery 
                                                  temperature.
            \item \textbf{Shutdown Voltage}: Electronics need a minimum voltage to operate correctly. If the battery voltage drops(steadily or suddenly) 
                                             below this threshold, the device will shutdown even if SoC is not zero. In this section, we'll assume that 
                                             if the battery voltage drops below 3.2V, the device will shutdown immediately.
        \end{itemize}

    \subsection{Temperature Behavior of Battery}
        Actual battery behavior is significantly influenced by temperature variations. The most common parameter affected by temperature is the internal 
        resistance. Using the Arrhenius' equation, we can model the temperature dependence of internal resistance as follows:
        \begin{equation}
            R = R_0 \cdot e^{\dfrac{E_a}{R_u T}}
        \end{equation}
        where $R_0$ is a reference resistance, $E_a$ is the activation energy, $R_u$ is the universal gas constant and $T_0$ is a reference temperature.
        
        To conduct the regression, rewrite the equation in linear form by taking the natural logarithm:
        \begin{equation}
            \ln R  = \frac{E_a}{R_u} \cdot \dfrac{1}{T} + \ln R_0
        \end{equation}

        The Samsung INR21700 30T 3Ah Li-ion Battery Dataset contains experimental data at various temperatures. Using this dataset again, we perform a 
        linear regression, yielding the following results:

        \begin{figure}[H]
            \centering
            \begin{subfigure}{0.3\textwidth}
                \includegraphics[width=\linewidth]{3d_surface_R0.png}
                \caption{$R_0$, $E_a = 17.47 kJ/mol$}
            \end{subfigure}
            \hfill
            \begin{subfigure}{0.3\textwidth}
                \includegraphics[width=\linewidth]{3d_surface_R1.png}
                \caption{$R_1$, $E_a = 37.24 kJ/mol$}
            \end{subfigure}
            \hfill
            \begin{subfigure}{0.3\textwidth}
                \includegraphics[width=\linewidth]{3d_surface_R2.png}
                \caption{$R_2$, $E_a = 15.09 kJ/mol$}
            \end{subfigure}
            \caption{3D Surface Fitting of Resistance with Temperature and SoC}
        \end{figure}

        If we consider the temperature effect, the simulation heatmap of discharge time from $100\%$ SoC under different power load and temperature will 
        be as follows: 
        \begin{figure}[H]
            \centering
            \includegraphics[width=0.6\linewidth]{depletion_time_heatmap.png}
            \caption{Heatmap of Discharge Time with Power Load and Temperature}
        \end{figure}

        It can be observed that battery behaves badly at low temperatures, and the discharge time decreases significantly as temperature drops. However,
        although high temperature improves battery short-term performance, it also accelerates battery degradation over time. Therefore, in practical 
        applications, maintaining an optimal temperature range is crucial for battery longevity and performance.

        %TODO: Battery degradation model with temperature

    \subsection{Battery in Different Work Loads}
        A smartphone is a complex embedded system whose total power consumption can be decomposed into the sum of contributions from its independent 
        subsystems. According to the \textbf{Principle of Power Superposition}, when individual modules share a power supply but operate relatively 
        independently, the total power draw can be expressed as a linear superposition of the power consumption of each module.

        Based on a functional module breakdown, we decompose the smartphone's power consumption into the following six primary sources:
        \begin{itemize}
            \item \textbf{Display Screen}: Backlight/OLED driver
            \item \textbf{CPU Computation}: Processor dynamic power
            \item \textbf{Network Communication}: WiFi/Cellular radio frequency (RF)
            \item \textbf{Location Service}: GPS reception and computation
            \item \textbf{Audio Playback}: Codec and speaker driver
            \item \textbf{System Mode}: Adjustment effects from power-saving/flight modes
        \end{itemize}

        Overall, the model can be written as:
        \begin{equation}
            \label{eq:P_total}
            P_{total} = P_{screen} + P_{CPU} + P_{network} + P_{GPS} + P_{audio} + P_{mode}
        \end{equation}

        By utilizing acquired smartphone state data, we estimate the parameters within these models, thereby obtaining a comprehensive power consumption 
        model for complex usage scenarios.  %TODO: Reference to dataset

        \subsubsection{Screen Power \texorpdfstring{$P_{screen}$}{P\_screen}}
        When the screen is on, it exhibits a base power draw plus a near-linear OLED power component (the modeling here assumes an OLED display):
        \begin{equation}
            \label{eq:P_screen}
            P_{screen} = \alpha_S \cdot S + \alpha_B \cdot S \cdot \frac{B}{255}
        \end{equation}
        where $S$ is a screen state indicator and $B$ is the brightness level (0--255).

        \subsubsection{CPU Power \texorpdfstring{$P_{CPU}$}{P\_CPU}}
        CPU power is the largest variable power source in a smartphone. According to the dynamic power formula for CMOS circuits:
        \begin{equation}
            P_{dynamic} = C \cdot V^2 \cdot f
        \end{equation}
        where $C$ is the load capacitance, $V$ the operating voltage, and $f$ the clock frequency.

        Modern processors employ \textbf{Dynamic Voltage and Frequency Scaling (DVFS)}, introducing a coupling between voltage and frequency. Empirical 
        studies suggest:
        \begin{equation}
            V \propto f^{0.5} \quad \Rightarrow \quad P \propto f^{2.5}
        \end{equation}

        Furthermore, modern ARM processors commonly adopt a \textbf{big.LITTLE heterogeneous architecture}, featuring separate big and small cores with 
        distinct performance and power characteristics. Additionally, actual CPU power depends on its utilization; idle and fully loaded states at the 
        same frequency exhibit significantly different power draws. Thus, we derive the following CPU power model:
        \begin{equation}
            \label{eq:P_CPU}
            P_{CPU} = \alpha_U \cdot U + \alpha_{big} \cdot \left(\frac{f_{big}}{f_{\max,big}}\right)^{2.5} + \alpha_{small} \cdot \left(\frac{f_{small}}{f_{\max,small}}\right)^{2.5}
        \end{equation}
        where $U$ represents CPU utilization, and $f_{big}$, $f_{small}$ are the operating frequencies of the big and small cores, normalized to their 
        respective maximum frequencies.

        \subsubsection{Network Power \texorpdfstring{$P_{network}$}{P\_network}}
        Power consumption in wireless communication stems primarily from the Radio Frequency (RF) front-end, especially the Power Amplifier (PA). 
        According to the Friis transmission equation, received power is inversely proportional to the square of the distance:
        \begin{equation}
            P_r = P_t \cdot G_t \cdot G_r \cdot \left(\frac{\lambda}{4\pi d}\right)^2
        \end{equation}

        Consequently, to maintain communication quality over distances up to several kilometers for cellular networks, relatively high transmit power is 
        required (typically $200 \sim 500 \text{mW}$). In contrast, WiFi communication with a router within tens of meters uses lower transmit power 
        (typically $50 \sim 100 \text{mW}$). Moreover, PA efficiency is typically only $30 \sim 40\%$, with substantial energy converted to heat, making
        network power a significant contributor to total consumption.

        Since WiFi and cellular networks are typically used exclusively in practice, our model uses a single indicator variable for network type:
        \begin{equation}
            \label{eq:P_network}
            P_{network} = \alpha_M \cdot M
        \end{equation}
        where $M$ indicates active network type.

        \subsubsection{GPS Power \texorpdfstring{$P_{GPS}$}{P\_GPS}}
        The GPS module continuously receives weak signals from multiple satellites and performs complex signal demodulation and position calculation. 
        Its power draw is relatively stable and depends mainly on whether it is active. We adopt a discrete model:
        \begin{equation}
            \label{eq:P_GPS}
            P_{GPS} = \alpha_G \cdot G
        \end{equation}
        where $G$ is a binary indicator for GPS activity.

        \subsubsection{Audio Power \texorpdfstring{$P_{audio}$}{P\_audio}}
        The audio module primarily involves compression/decompression processing via an audio DSP, digital-to-analog conversion (DAC), and driving 
        speakers or headphones. We simplify the modeling to a binary indicator for audio playback activity:
        \begin{equation}
            \label{eq:P_audio}
            P_{audio} = \alpha_A \cdot A
        \end{equation}

        \subsubsection{System Mode Power Adjustment \texorpdfstring{$P_{mode}$}{P\_mode}}
        Modern smartphones offer various system modes for power management, such as Power Saving Mode and Flight Mode. However, most of their energy-saving effects are achieved by altering other parameters (e.g., reducing CPU frequency, limiting network activity), which are already captured by other terms in the model. Therefore, $P_{mode}$ represents additional energy-saving mechanisms. Power Saving Mode may disable unnecessary sensors, reduce refresh rates, and optimize memory management, while Flight Mode directly powers down RF modules. The model is:
        \begin{equation}
            \label{eq:P_mode}
            P_{mode} = \alpha_E \cdot E + \alpha_F \cdot F
        \end{equation}
        where $E$ and $F$ are indicators for Power Saving Mode and Flight Mode, respectively. Expectedly, $\alpha_E \leq 0$ and $\alpha_F \leq 0$.

        \subsubsection{Parameter Estimation and Validation}
        We performed physically constrained regression using the L-BFGS-B optimizer on normalized and binarized data, enforcing non-negative coefficients 
        for power components and non-positive coefficients for power-saving modes. Substituting the estimated parameters yields the final empirical model:
        \begin{equation}
            \label{eq:final_model}
            \boxed{\begin{aligned}
                P_{total} &= 0.250S + 0.615S\frac{B}{255} + 0.860U + 1.125f_{big}^{2.5} + 0.650f_{small}^{2.5} \\
                &\quad + 0.696M + 0.040G + 0.397A - 0.068E - 0.028F
            \end{aligned}}
        \end{equation}

        Subsequently, we can analyze the contribution rate of each parameter and evaluate the model's performance. The following figure illustrates the fitting results and validation:

        \begin{figure}[H]
            \centering
            \begin{subfigure}{0.48\textwidth}
                \includegraphics[width=\linewidth]{power_model_coefficients.png}
            \end{subfigure}
            \hfill
            \begin{subfigure}{0.48\textwidth}
                \includegraphics[width=\linewidth]{power_model_contribution.png}
            \end{subfigure}
            
            \vspace{0.5em}
            
            \begin{subfigure}{0.48\textwidth}
                \includegraphics[width=\linewidth]{power_model_prediction.png}
            \end{subfigure}
            \hfill
            \begin{subfigure}{0.48\textwidth}
                \includegraphics[width=\linewidth]{power_model_residuals.png}
            \end{subfigure}
            \caption{Power consumption model fitting results and validation}
            \label{fig:power_model_fitting}
        \end{figure}

        Results demonstrate that the white-box power model constructed based on the real-world smartphone dataset achieves satisfactory prediction performance, with an $R^2$ of 0.606, an MAE of 0.355 W, and an RMSE of 0.461 W. The model is able to explain over 60\% of the variance in power consumption. Among the factors analyzed, CPU usage and data network activity exhibit the most significant impact on power draw, whereas GPS consumption and the energy-saving effects of the two included modes show relatively minor influence.

        \subsubsection{Usage Scenario Simulation}
        The model allows us to estimate the smartphone's power consumption under various typical usage scenarios. Here are some typical scenarios and 
        their estimated power consumption:
        \begin{table}[H]
            \centering
            \label{tab:power_consumption}
            \resizebox{\columnwidth}{!}{
                \begin{tabular}{@{}lccccccccc@{}}
                \toprule
                \textbf{Scenario} & \textbf{Screen} & \textbf{Brightness} & \textbf{CPU} & \textbf{Big Core} & \textbf{Small Core} & \textbf{Mobile} & \textbf{GPS} & \textbf{Audio} & \textbf{Total Power} \\
                & \textbf{(S)} & \textbf{(B/255)} & \textbf{Usage} & \textbf{Freq} & \textbf{Freq} & \textbf{Network} & & & \textbf{(W)} \\
                \midrule
                Standby        & Off & 0\%   & 10\% & 10\%  & 10\%  & No  & Off & Off & 0.09 \\
                Web Browsing   & On  & 50\%  & 50\% & 30\%  & 30\%  & No  & Off & Off & 1.08 \\
                Video Streaming& On  & 71\%  & 40\% & 40\%  & 30\%  & No  & Off & On  & 1.57 \\
                Navigation     & On  & 100\% & 50\% & 50\%  & 40\%  & Yes & On  & On  & 2.69 \\
                Gaming         & On  & 100\% & 90\% & 100\% & 100\% & Yes & Off & On  & 4.51 \\
                \bottomrule
                \end{tabular}
            }
            \caption{Power Consumption Under Different Usage Scenarios}
        \end{table}

        %TODO: Conclusions from usage scenarios

    \subsection{Heat Transfer Model in High-Performance Scenarios}
        In this chapter, we discuss how the heat that battery generates affacts the temperature of battery itself, and in turn impacts its own 
        performance.

        %TODO: Assumptions about heat transfer model

        Let $Q$ be the internal heat power, $h$ be the convective heat transfer coefficient, $A$ be the surface area of the battery(assumed to be the area
        of the smartphone), $C$ be the heat capacity of the battery, $T$ be the battery temperature and $T_{env}$ be the environmental temperature(assumed
        constant). With Newton's law of cooling, we have:
        \begin{equation}
            \label{Law of cooling}
            Q = 2Ah (T - T_{env}) + C \frac{dT}{dt}
        \end{equation}
        
        $Q$ can be divided to:
        \begin{equation}
            Q = Q_{battery} + Q_{processor} + Q_{other}
        \end{equation}

        $Q_{battery}$ can be further expressed with the Joule's law:
        \begin{equation}
            Q_{battery} = I^2 R_{inter} = I^2 (R_0 + R_1 + R_2)
        \end{equation}

        $Q_{processor}$ often is the major part of heat generation, because processors are often energy-intensive and almost all consumed energy is 
        converted to heat. Here we model $Q_{processor}$ as a part of total power consumption of the device:
        \begin{equation}
            Q_{processor} = \eta P_{total}
        \end{equation}
        in which $\eta$ is some coefficient.

        And the other components' heat generation $Q_{other}$ can be assumed to be a constant for simplicity.

        Considering that $I$ and $R$ is time-dependent, equation \eqref{Law of cooling} is a first-order linear ODE. The solution is:
        \begin{equation}
            \label{Relation between T and t}
            T = T_{env} + \frac{1}{C} \int_0^t{e^{-\dfrac{2Ah(t - s)}{C}} Q(s) \mathrm{d}s}
        \end{equation}

        Substitute equation \eqref{Relation between T and t} into the Arrhenius' equation, we can get the relation between internal resistance and time,
        and thus analyze how temperature affects battery performance over time in high-performance scenarios.

        The simulation here shows the heatmap of discharge time and max temperature under different power load and environmental temperature. It assumes 
        if the battery temperature exceeds $50\mathrm{^\circ C}$, the device will shutdown to protect the battery. In reality, smartphones often decreases processor
        frequency to reduce heat generation when temperature is too high, but for simplicity we assume an immediate shutdown here.

        \begin{figure}[H]
            \centering
            \begin{subfigure}{0.45\textwidth}
                \includegraphics[width=\linewidth]{depletion_time_thermal_heatmap.png}
            \end{subfigure}
            \hfill
            \begin{subfigure}{0.45\textwidth}
                \includegraphics[width=\linewidth]{max_temperature_thermal_heatmap.png}
            \end{subfigure}
            \caption{Heatmap of Discharge Time and Max Temperature}
        \end{figure}

        In the simulation we set $C = 160 \mathrm{J/K}$, $A = 200 \mathrm{cm}^2$, $h = 5 \mathrm{W/(m^2 \cdot K)}$, $\eta = 0.5$ and 
        $Q_{other} = 0.8 \mathrm{W}$.

        It can be observed that high temperature scenarios are quite dangerous for battery operation. Heavy work loads in $30 \mathrm{^\circ C}$ or higher
        can easily lead to battery overheating.