\section{Actual Battery}
    \subsection{Assumptions}
        An actual battery is much more complex than the idealized model presented in the previous section. In this section, we will consider more factors
        that affect the performance and behavior of real batteries:
        \begin{itemize}
            \item \textbf{Temperature}: The performance of a battery can vary significantly with temperature. At low temperatures, the internal resistance
                                        increases, leading to reduced capacity and power output. Conversely, high temperatures can enhance performance but
                                        may also accelerate degradation.
            \item \textbf{Complex Power Profile}: Real batteries often experience varying power demands, which can affect their efficiency and lifespan.
                                                  High performance demands can lead to increased heat generation, which int turns affect battery 
                                                  temperature.
            \item \textbf{Shutdown Voltage}: Electronics need a minimum voltage to operate correctly. If the battery voltage drops(steadily or suddenly) 
                                             below this threshold, the device will shutdown even if SoC is not zero. In this section, we'll assume that 
                                             if the battery voltage drops below 3.2V, the device will shutdown immediately.
        \end{itemize}

    \subsection{Temperature Behavior of Battery}
        Actual battery behavior is significantly influenced by temperature variations. The most common parameter affected by temperature is the internal 
        resistance. Using the Arrhenius' equation, we can model the temperature dependence of internal resistance as follows:
        \begin{equation}
            R = R_0 \cdot e^{\dfrac{E_a}{R_u T}}
        \end{equation}
        where $R_0$ is a reference resistance, $E_a$ is the activation energy, $R_u$ is the universal gas constant and $T_0$ is a reference temperature.
        
        To conduct the regression, rewrite the equation in linear form by taking the natural logarithm:
        \begin{equation}
            \ln R  = \frac{E_a}{R_u} \cdot \dfrac{1}{T} + \ln R_0
        \end{equation}

        The Samsung INR21700 30T 3Ah Li-ion Battery Dataset contains experimental data at various temperatures. Using this dataset again, we perform a 
        linear regression, yielding the following results:

        \begin{figure}[H]
            \centering
            \begin{subfigure}{0.3\textwidth}
                \includegraphics[width=\linewidth]{3d_surface_R0.png}
                \caption{$R_0$, $E_a = 17.47 kJ/mol$}
            \end{subfigure}
            \hfill
            \begin{subfigure}{0.3\textwidth}
                \includegraphics[width=\linewidth]{3d_surface_R1.png}
                \caption{$R_1$, $E_a = 37.24 kJ/mol$}
            \end{subfigure}
            \hfill
            \begin{subfigure}{0.3\textwidth}
                \includegraphics[width=\linewidth]{3d_surface_R2.png}
                \caption{$R_2$, $E_a = 15.09 kJ/mol$}
            \end{subfigure}
            \caption{3D Surface Fitting of Resistance with Temperature and SoC}
        \end{figure}

        If we consider the temperature effect, the simulation heatmap of discharge time from $100\%$ SoC under different power load and temperature will 
        be as follows: 
        \begin{figure}[H]
            \centering
            \includegraphics[width=0.6\linewidth]{depletion_time_heatmap.png}
            \caption{Heatmap of Discharge Time with Power Load and Temperature}
        \end{figure}

        It can be observed that battery behaves badly at low temperatures, and the discharge time decreases significantly as temperature drops. However,
        although high temperature improves battery short-term performance, it also accelerates battery degradation over time. Therefore, in practical 
        applications, maintaining an optimal temperature range is crucial for battery longevity and performance.

        %TODO: Battery degradation model with temperature

    \subsection{Battery in Different Work Loads}

    \subsection{Heat Transfer Model in High-Performance Scenarios}
        In this chapter, we discuss how the heat that battery generates affacts the temperature of battery itself, and in turn impacts its own 
        performance.

        %TODO: Assumptions about heat transfer model

        Let $Q$ be the internal heat power, $h$ be the convective heat transfer coefficient, $A$ be the surface area of the battery(assumed to be the area
        of the smartphone), $C$ be the heat capacity of the battery, $T$ be the battery temperature and $T_{env}$ be the environmental temperature(assumed
        constant). With Newton's law of cooling, we have:
        \begin{equation}
            \label{Law of cooling}
            Q = 2Ah (T - T_{env}) + C \frac{dT}{dt}
        \end{equation}
        
        $Q$ can be divided to:
        \begin{equation}
            Q = Q_{battery} + Q_{processor} + Q_{other}
        \end{equation}

        $Q_{battery}$ can be further expressed with the Joule's law:
        \begin{equation}
            Q_{battery} = I^2 R_{inter} = I^2 (R_0 + R_1 + R_2)
        \end{equation}

        $Q_{processor}$ often is the major part of heat generation, because processors are often energy-intensive and almost all consumed energy is 
        converted to heat. Here we model $Q_{processor}$ as a part of total power consumption of the device:
        \begin{equation}
            Q_{processor} = \eta P_{total}
        \end{equation}
        in which $\eta$ is some coefficient.

        And the other components' heat generation $Q_{other}$ can be assumed to be a constant for simplicity.

        Considering that $I$ and $R$ is time-dependent, equation \eqref{Law of cooling} is a first-order linear ODE. The solution is:
        \begin{equation}
            \label{Relation between T and t}
            T = T_{env} + \frac{1}{C} \int_0^t{e^{-\dfrac{2Ah(t - s)}{C}} Q(s) \mathrm{d}s}
        \end{equation}

        Substitute equation \eqref{Relation between T and t} into the Arrhenius' equation, we can get the relation between internal resistance and time,
        and thus analyze how temperature affects battery performance over time in high-performance scenarios.

        The simulation here shows the heatmap of discharge time and max temperature under different power load and environmental temperature. It assumes 
        if the battery temperature exceeds $50\mathrm{^\circ C}$, the device will shutdown to protect the battery. In reality, smartphones often decreases processor
        frequency to reduce heat generation when temperature is too high, but for simplicity we assume an immediate shutdown here.

        \begin{figure}[H]
            \centering
            \begin{subfigure}{0.45\textwidth}
                \includegraphics[width=\linewidth]{depletion_time_thermal_heatmap.png}
            \end{subfigure}
            \hfill
            \begin{subfigure}{0.45\textwidth}
                \includegraphics[width=\linewidth]{max_temperature_thermal_heatmap.png}
            \end{subfigure}
            \caption{Heatmap of Discharge Time and Max Temperature}
        \end{figure}

        In the simulation we set $C = 160 \mathrm{J/K}$, $A = 200 \mathrm{cm}^2$, $h = 5 \mathrm{W/(m^2 \cdot K)}$, $\eta = 0.5$ and 
        $Q_{other} = 0.8 \mathrm{W}$.

        It can be observed that high temperature scenarios are quite dangerous for battery operation. Heavy work loads in $30 \mathrm{^\circ C}$ or higher
        can easily lead to battery overheating.
 
    \subsection{Answer to Question 1}

    \subsection{Answer to Question 2}