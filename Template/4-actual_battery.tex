\section{Actual Battery}
    \subsection{Assumptions}
        An actual battery is much more complex than the idealized model presented in the previous section. In this section, we will consider more factors
        that affect the performance and behavior of real batteries:
        \begin{itemize}
            \item \textbf{Temperature}: The performance of a battery can vary significantly with temperature. At low temperatures, the internal resistance
                                        increases, leading to reduced capacity and power output. Conversely, high temperatures can enhance performance but
                                        may also accelerate degradation.
            \item \textbf{Complex Power Profile}: Real batteries often experience varying power demands, which can affect their efficiency and lifespan.
                                                  High performance demands can lead to increased heat generation, which int turns affect battery 
                                                  temperature.
            \item \textbf{Shutdown Voltage}: Electronics need a minimum voltage to operate correctly. If the battery voltage drops(steadily or suddenly) 
                                             below this threshold, the device will shutdown even if SoC is not zero.
        \end{itemize}

    \subsection{Temperature Behavior of Battery}
        Actual battery behavior is significantly influenced by temperature variations. The most common parameter affected by temperature is the internal 
        resistance. Using the Arrhenius' equation, we can model the temperature dependence of internal resistance as follows:
        \begin{equation}
            R = R_0 \cdot e^{ - \dfrac{E_a}{R_u T}}
        \end{equation}
        where $R_0$ is a reference resistance, $E_a$ is the activation energy, $R_u$ is the universal gas constant and $T_0$ is a reference temperature.
        
        To conduct the regression, rewrite the equation in linear form by taking the natural logarithm:
        \begin{equation}
            \ln R  = - \frac{E_a}{R_u} \cdot \dfrac{1}{T} + \ln R_0
        \end{equation}

        The Samsung INR21700 30T 3Ah Li-ion Battery Dataset contains experimental data at various temperatures. Using this dataset again, we perform a 
        linear regression, yielding the following results:

        \begin{figure}[H]
            \centering
            \begin{subfigure}{0.3\textwidth}
                \includegraphics[width=\linewidth]{3d_surface_R0.png}
                \caption{$R_0$, $E_a = 17.47 kJ/mol$}
            \end{subfigure}
            \hfill
            \begin{subfigure}{0.3\textwidth}
                \includegraphics[width=\linewidth]{3d_surface_R1.png}
                \caption{$R_1$, $E_a = 37.24 kJ/mol$}
            \end{subfigure}
            \hfill
            \begin{subfigure}{0.3\textwidth}
                \includegraphics[width=\linewidth]{3d_surface_R2.png}
                \caption{$R_2$, $E_a = 15.09 kJ/mol$}
            \end{subfigure}
            \caption{3D Surface Fitting of Resistance with Temperature and SoC}
        \end{figure}

    \subsection{Battery in Different Work Loads}

    \subsection{Heat Transfer Model in High-Performance Scenarios}
        In this chapter, we discuss how the heat that battery generates affacts the temperature of battery itself, and in turn impacts its own 
        performance.

        %TODO: Assumptions about heat transfer model

        Let $Q$ be the internal heat power, $h$ be the convective heat transfer coefficient, $A$ be the surface area of the battery(assumed to be the area
        of the smartphone), $C$ be the heat capacity of the battery, $T$ be the battery temperature and $T_{env}$ be the environmental temperature(assumed
        constant). With Newton's law of cooling, we have:
        \begin{equation}
            \label{Law of cooling}
            Q = h A (T - T_{env}) + C \frac{dT}{dt}
        \end{equation}
        
        $Q$ can be expressed by the Joule's law as:
        \begin{equation}
            Q = I^2 R_{inter} = I^2 (R_0 + R_1 + R_2)
        \end{equation}

        Considering that $I$ and $R$ is time-dependent, equation \eqref{Law of cooling} is a first-order linear ODE. The solution is:
        \begin{equation}
            \label{Relation between T and t}
            T = T_{env} + \frac{1}{C} \int_0^t{e^{-\dfrac{2Ah(t - s)}{C}} Q(s) \mathrm{d}s}
        \end{equation}

        Substitute equation \eqref{Relation between T and t} into the Arrhenius equation, we can get the relation between internal resistance and time,
        and thus analyze how temperature affects battery performance over time in high-performance scenarios.
 
    \subsection{Answer to Question 1}

    \subsection{Answer to Question 2}