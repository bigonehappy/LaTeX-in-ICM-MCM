\section{Introduction}
    \subsection{Problem Background}
        Smartphone has become an indispensable part of our daily lives. What plays a key role in user experience and smartphones' design is the 
        development of smartphone battery. From where consumers stand, we don't want to be anxious as battery's draining behaviour is affected by a 
        tremendous amount of factors, like the processer load and the ambient temperature. Still, as manufacturers, we need to provide the most accurate
        prediction of the as a competitive edge, and most importantly, to adapt subsequent measures to increase battery's lifespan, which helps a lot in 
        environmental sustainability. All of these are calling for more-detailed physical models and optimized algorithm predictions.	
        \begin{figure}[H]
            \centering
            \includegraphics[width=0.4\textwidth]{animate.png}
        \end{figure}
    \subsection{Restatement of the Problem}
        Based on the background, our aim is to establish an algorithm from the very physical essence of battery theory and user's data, through which we
        could finally help users to predict the time-to-empty under different circumstances and adapt adequate optimizations. So we need to address:
        \begin{enumerate}
            \item \textbf{Continuous-Time Model}: Deriving a set of differential equations from lithium battery's principle, establishing an ideal and 
                                                  then an actual model that returns the $SOC$ as a function of time, and using collected data to validate 
                                                  and support them.
            \item \textbf{Time-to-Empty Predictions}: Approximating the time-to-empty from different initial charge level, and introduce complex 
                                                      conditions to find the differences.
            \item \textbf{Sensitivity Analysis}: Introducing fluctuations of usage patterns or making changes to some flexible parameters or the 
                                                 assumptions to examine the model for lithium batteries is solid.
            \item \textbf{Recommendations}: Finding out optimization strategies of improving battery life, considering battery aging's impact on effective
                                            capacity's reduction, and trying to generalize to other portable devices. 
        \end{enumerate}
    \subsection{Our Work}
        \begin{figure}[H]
            \centering
            \includegraphics[width=0.8\textwidth]{flowchart.png}
            \caption{Flowchart of Our Work}
        \end{figure}