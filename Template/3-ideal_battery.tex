\section{Ideal Battery}
    \subsection{Assumptions}
        To consider an simplist ideal battery model, the following assumptions are made:
        \begin{itemize}
            \item The battery is kept at a constant environment, so the temperature effects are neglected.
            \item The battery has no energy loss during charging and discharging.
            \item The battery has infinite cycle life, so the degradation effects are neglected and all the parameters are constant.
            \item The Open Circuit Voltage (OCV) is only related to the $SOC$, so we can express the $SOC$ as a functionof $U_{OC}$:
                  \begin{equation}
                    SOC = f(U_{OC})
                  \end{equation}
                  in which $f(\cdot)$ can be obtained through curve fitting based on experimental data.
            \item The battery behavior is same for charging and discharging. For smartphones, which mostly use \ce{LiCoO2} batteries that has little 
                  or no hysteresis, this assumption is reasonable.
        \end{itemize}

    \subsection{Thevenin Model}
        A Li-ion battery system can be extremely complex, involving electrochemical, thermal and mechanical processes. However, for system-level studies,
        an equivalent circuit model is often used to represent the battery behavior~\cite{ecm_model}. The 2-order Thevenin model is a widely used 
        equivalent circuit model that captures the dynamic response of the battery voltage during charge and discharge cycles. The model is described
        as follows:

        \begin{figure}[H]
            \centering
            \includegraphics[width=0.6\textwidth]{Thevenin Model.png}
            \caption{Thevenin Model Circuit}
        \end{figure}

        In electrochemistry, the polarization effects are often divided into two main categories: electrochemical polarization and concentration
        polarization. The former is caused by the charge transfer resistance at the electrode/electrolyte interface, while the latter is due to the mass
        transport limitations within the electrode materials. In the Thevenin model, these two polarization effects are represented by two RC networks: 
        $R_1, C_1$ for electrochemical polarization and $R_2, C_2$ for concentration polarization.

        \begin{figure}[H]
            \centering
            \includegraphics[width=0.8\textwidth]{battery.png}
            \caption{Battery and Its Model Equivalent Circuit}
        \end{figure}

        The relationship between $U_{OC}$ and other parameters in the Thevenin model can be expressed with Kirchhoff's laws:
        \begin{equation}
            \label{Thevenin_model_KVL}
            U_{OC} = U_t + I R_0 + U_1 + U_2
        \end{equation}
        in which $U_t$ is the terminal voltage.

        For $U_1, U_2$ we have:
        \begin{equation}
            \label{differential_equation_polarization}
            I = C_1 \frac{\mathrm{d}U_1}{\mathrm{d}t} + \frac{U_1}{R_1} = C_2 \frac{\mathrm{d}U_2}{\mathrm{d}t} + \frac{U_2}{R_2}
        \end{equation}

        Differentiateing the $SOC$'s definition with respect to time, we get: %TODO: SOC definition reference
        \begin{equation}
            \label{differential_equation_SOC_with_current}
            \frac{\mathrm{d}(SOC)}{\mathrm{d}t} = -\frac{I}{Q_{max}}
        \end{equation}
        where $Q_{max}$ is the maximum capacity of the battery.

        Combining the above equations, we can derive the complete Thevenin model in matrix form:
        \begin{equation}
            \label{Thevenin_model_matrix_form}
            \renewcommand{\arraystretch}{2}
            \frac{\mathrm{d}}{\mathrm{d}t}
            \begin{bmatrix}
                SOC  \\
                U_1  \\
                U_2
            \end{bmatrix}
            =
            \begin{bmatrix}
                0                      & 0                        & 0 \\
                0                      &  -\dfrac{1}{R_{1} C_{1}} & 0 \\
                0                      & 0                        & -\dfrac{1}{R_{2} C_{2}}
            \end{bmatrix}
            \begin{bmatrix}
                SOC  \\
                U_1 \\
                U_2
            \end{bmatrix}
            +
            \begin{bmatrix}
                -\dfrac{1}{Q_{max}}    \\
                \dfrac{1}{C_1}    \\
                \dfrac{1}{C_2}
            \end{bmatrix}
            I
        \end{equation}
        where $R_0, R_1, R_2, C_1, C_2, Q_{max}$ are the model parameters that need to be estimated.

        However, if the battery is in an open-circuit state, the structure of the circuit will change. In that case, since $I = 0$, there is no voltage 
        drop across $R_0$, so the terminal voltage $U_t$ is expressed by:
        \begin{equation}
            \label{Open-circuit terminal voltage}
            U_t = U_{OC} - U_1 - U_2
        \end{equation}

        Both of the capactors $C_1, C_2$ will slowly discharge through their respective resistors $R_1, R_2$, expressed as:
        \begin{equation}
            \label{Open-circuit RC equation}
            \frac{\mathrm{d}U_1}{\mathrm{d}t} = -\frac{U_1}{R_1 C_1}, \quad
            \frac{\mathrm{d}U_2}{\mathrm{d}t} = -\frac{U_2}{R_2 C_2}
        \end{equation}

        Combining \eqref{Open-circuit terminal voltage} and \eqref{Open-circuit RC equation}, we can get the analytic solution:
        \begin{equation}
            \label{Open-circuit terminal voltage solution}
            U_t = U_{OC} - U_{1, init} e^{-\dfrac{t}{R_1 C_1}} - U_{2, init} e^{-\dfrac{t}{R_2 C_2}}
        \end{equation}
        where $U_{1, init}, U_{2, init}$ are the initial voltages across $C_1, C_2$ at the start of the open-circuit state.

    \subsection{Parameter Estimation}
        The Hybrid Pulse Power Characterization (HPPC) test is commonly used for this purpose. The test is conducted with steps as follows:
        \begin{enumerate}
            \item The battery is first fully charged to $100\%$ $SOC$ in ways the manufacturer recommends.
            \item After resting for a certain period(e.g. 1 hour), the battery is discharged with a constant current pulse (e.g., $0.5C$) for a short 
                  duration (e.g., $10s$). The voltage response is recorded.
            \item Then discharge the battery to another selected $SOC$ point (e.g., $90\%$).
            \item Repeat steps 2 and 3 until the battery reaches a low $SOC$ point (e.g., $10\%$) or the maximum discharge limit the manufacturer specifies.
            \item If needed, repeat similar steps for charging pulses.
        \end{enumerate}
        The resting step between pulses allows the exponential term in equation \eqref{Open-circuit terminal voltage solution} to decay to zero, so 
        accurate $U_{OC}$ can be obtained.

        With HPPC data, the model parameters can be estimated through curve fitting techniques. The process are as follows:
        \begin{itemize}
            \item \textbf{Estimate $R_0$}: 
                    The instantaneous voltage drop at the start of each pulse can be used to estimate the internal resistance $R_0$, at which point 
                    the capacitive effects are negligible.
                    \begin{equation}
                        R_0 = \frac{\Delta U_{instant}}{I_{pulse}}
                    \end{equation}
            \item \textbf{Estimate $R_1, C_1$ and $R_2, C_2$}:
                    Solve the polarization equation \eqref{differential_equation_polarization}, we can get the polarization voltage response:
                    \begin{equation}
                        U_{polar} = IR + (U_{init} - IR) e^{-\dfrac{t}{R C}}
                    \end{equation}
                    where $U_{init}$ is the voltage at the start of the pulse.

                    Without loss of generality, let $R_1 C_1 \leq R_2 C_2$, so that the faster electrochemical polarization are represented by $R_1, C_1$ 
                    and the slower concentration polarization effects are represented by $R_2, C_2$. Then substitute them into \eqref{Thevenin_model_KVL},
                    we have:
                    \begin{equation}
                        \label{R_C_Estimation}
                        U_t = U_{OC} - I(R_0 + R_1 + R_2) - (U_{1, init} - IR_1) e^{-\dfrac{t}{R_1 C_1}} - (U_{2, init} - IR_2) e^{-\dfrac{t}{R_2 C_2}}
                    \end{equation}

                    By least squares fitting of \eqref{R_C_Estimation} to the voltage response data during each pulse, we can estimate the values of
                    $R_1, C_1$ and $R_2, C_2$ at different $SOC$ points.
            \item \textbf{Estimate $Q_{max}$}: 
                    The maximum capacity $Q_{max}$ can be estimated by integrating the current over the full discharge cycle:
                    \begin{equation}
                        Q_{max} = \int_{t_0}^{t_f} I(t) \mathrm{d}t
                    \end{equation}
                    where $t_0$ and $t_f$ are the start and end times of the discharge cycle.

                    Most of the time this value is provided by the manufacturer.
        \end{itemize}

        We use the Samsung INR21700 30T 3Ah Li-ion Battery Dataset~\cite{samsung_dataset} to demonstrate the parameter estimation process. The fitting results are shown as follows:
        \begin{figure}[H]
            \centering

            \begin{subfigure}{0.45\textwidth}
                \includegraphics[width=\linewidth]{fitting_OCV.png}
                \label{fig:fitting_OCV}
            \end{subfigure} 
            \hfill
            \begin{subfigure}{0.45\textwidth}
                \includegraphics[width=\linewidth]{fitting_R0.png}
                \label{fig:fitting_R0}
            \end{subfigure} 

        \end{figure}
        \begin{figure}[H]
            \centering

            \begin{subfigure}{0.45\textwidth}
                \includegraphics[width=\linewidth]{fitting_C1.png}
                \label{fig:fitting_C1}
            \end{subfigure} 
            \hfill
            \begin{subfigure}{0.45\textwidth}
                \includegraphics[width=\linewidth]{fitting_R1.png}
                \label{fig:fitting_R1}
            \end{subfigure} 
                 
            \begin{subfigure}{0.45\textwidth}
                \includegraphics[width=\linewidth]{fitting_C2.png}
                \label{fig:fitting_C2}
            \end{subfigure} 
            \hfill
            \begin{subfigure}{0.45\textwidth}
                \includegraphics[width=\linewidth]{fitting_R2.png}
                \label{fig:fitting_R2}
            \end{subfigure} 
            \caption{Fitting results of the parameter estimation process.}
            \label{fig:parameter_estimation_results}
        \end{figure}

        6-degree polynomial is used to fit the OCV-$SOC$ curve, cubic functions are used to fit the $C_1, C_2$, and double exponential functions 
        $y = A_1 e^{B_1 x} + A_2 e^{B_2 x} + C$ are used to fit the $R_0, R_1, R_2$ curves. 

        As the fitting result shows, the Ohm resistance $R_0$ increases significantly when $SOC$ is low, which may be the main reason for voltage drop and 
        device shutdown. Later we will verify this conclusion. Polarization resistances $R_1, R_2$ also increase when $SOC$ is low, indicating the 
        battery's internal electrochemical processes are hindered. And Polarization capacitances $C_1, C_2$ decrease when $SOC$ is low, meaning that the 
        battery's ability to respond to load changes is weakened.

        Noticed that $C_1, C_2$ drop to zero when $SOC$ is low, which means the Thevenin model is not valid in that region. However, the smartphone will 
        actually shutdown before the battery reaches such low $SOC$, because the output voltage will be too low to power the device.
    \subsection{Simulations}
        With proper model parameters, we can do numerical simulations of the battery behavior under different load profiles. 2 simplified load profiles
        are considered here: 
        \begin{itemize}
            \item Constant current;
            \item Constant power.
        \end{itemize}

        \subsubsection{Constant current}
            Since the current is constant, equation \eqref{differential_equation_SOC_with_current} can be directly integrated to get the $SOC$ at time $t$:
            \begin{equation}
                SOC(t) = SOC(0) - \frac{I t}{Q_{max}}
            \end{equation}
            
            Then perform Rk45 algorithm to the polarization voltages $U_1, U_2$ with initial conditions $U_1(0) = U_2(0) = 0$, we can get the final result
            as follows.

            \begin{figure}[H]
            \centering

            \begin{subfigure}{0.45\textwidth}
                \includegraphics[width=\linewidth]{ci_SOC_profile.png}
                \label{fig:SOC_profile_constant_current}
            \end{subfigure} 
            \hfill
            \begin{subfigure}{0.45\textwidth}
                \includegraphics[width=\linewidth]{ci_voltage_profile.png}
                \label{fig:voltage_profile_constant_current}
            \end{subfigure} 

            \caption{Simulation results under constant current load profile}
        \end{figure}

        It can be seen that although the $SOC$ decreases linearly under constant current load, the terminal voltage decreases steadily at first, then 
        suddenly drops to a "dead" value, which makes the device shutdown. The phenomenon is consistent with our daily experience.

        \subsubsection{Constant power}
            The output power
            \begin{equation}
                P = U_t I
            \end{equation}
            is constant. In this case, analytic solutions are impossible, so we just use the Rk45 algorithm:

            \begin{figure}[H]
                \centering
                \begin{subfigure}{0.45\textwidth}
                    \includegraphics[width=\linewidth]{cp_SOC_profile.png}
                    \label{fig:SOC_profile_constant_power}
                \end{subfigure} 
                \hfill
                \begin{subfigure}{0.45\textwidth}
                    \includegraphics[width=\linewidth]{cp_voltage_profile.png}
                    \label{fig:voltage_profile_constant_power}
                \end{subfigure}
                \caption{Simulation results under constant power load profile}
            \end{figure}

            Similar phenomena can be observed under constant power load profile. However, the terminal voltage drops more rapidly compared to the
            constant current case, indicating that constant power loads are more stressful to the battery.