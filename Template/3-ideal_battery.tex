\section{Ideal Battery}
    \subsection{Assumptions}
        To consider an simplist ideal battery model, the following assumptions are made:
        \begin{itemize}
            \item The battery is kept at a constant environment, so the temperature effects are neglected.
            \item The battery has no energy loss during charging and discharging.
            \item The battery has infinite cycle life, so the degradation effects are neglected and all the parameters are constant.
        \end{itemize}

    \subsection{Thevenin Model}
        A Li-ion battery system can be extremely complex, involving electrochemical, thermal and mechanical processes. However, for system-level studies,
        an equivalent circuit model is often used to represent the battery behavior. The Thevenin model is a widely used equivalent circuit model that 
        captures the dynamic response of the battery voltage during charge and discharge cycles. The model is described as follows:

        % \begin{figure}[H]
        %     \centering
        %     \includegraphics[width=0.5\textwidth]{Thevenin_model.png}   % TODO: figure
        %     \caption{Thevenin Model of a Li-ion Battery}
        %     \label{fig:Thevenin_model}
        % \end{figure}

        According to experimental experiences, the SoC is positively correlated with the open-circuit voltage $U_{OC}$, so we can express SoC as a function
        of $U_{OC}$:
        \begin{equation}
            SoC = f(U_{OC})
        \end{equation}
        in which $f(\cdot)$ can be obtained through curve fitting based on experimental data.

        The relationship between $U_{OC}$ and other parameters in the Thevenin model can be expressed with Kirchhoff's laws:
        \begin{equation}
            U_{OC} = U_t + I R_{0} + U_{1}
        \end{equation}
        in which $U_t$ is the terminal voltage.

        For $U_1$ we have:
        \begin{equation}
            I = C_{1} \frac{dU_{1}}{dt} + \frac{U_{1}}{R_{1}}
        \end{equation}

        Differentiateing the SoC's definition with respect to time, we get: %TODO: SoC definition reference
        \begin{equation}
            \frac{d(SoC)}{dt} = -\frac{I}{Q_{max}}
        \end{equation}
        where $Q_{max}$ is the maximum capacity of the battery.

        Combining the above equations, we can derive the complete Thevenin model in matrix form:
        \begin{equation}
            \frac{d}{dt}
            \begin{bmatrix}
                SoC  \\
                U_{1}
            \end{bmatrix}
            =
            \begin{bmatrix}
                0                      & 0                     \\
                0                      & \displaystyle -\frac{1}{R_{1} C_{1}}
            \end{bmatrix}
            \begin{bmatrix}
                SoC  \\
                U_{1}
            \end{bmatrix}
            +
            \begin{bmatrix}
                \displaystyle -\frac{1}{Q_{max}}    \\
                \displaystyle  \frac{1}{C_{1}}
            \end{bmatrix}
            I
        \end{equation}
    \subsection{Simulations}

    \subsection{Result}