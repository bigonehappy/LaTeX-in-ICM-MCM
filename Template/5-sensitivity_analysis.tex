\section{Sensitivity Analysis}
    \subsection{Analysis of Basic Electrical Parameters}
        Smartphone batteries varies in capacity, voltage, and internal resistance. To understand how these variations affect the performance of the system,
        we conducted a sensitivity analysis by varying each parameter in equation \eqref{Thevenin_model_matrix_form} within realistic ranges.

        \subsubsection{Time-Varying Properties}
        Too much simplification leads to lose of actual behavior. For example, setting $R_0, R_1, R_2, C_1, C_2$ to constants rather than parameters 
        varies with SoC can result in inaccurate predictions. Here is the error result if we simply set them as sample mean values:
        \begin{figure}[H]
            \centering
            \includegraphics[width=0.6\textwidth]{voltage_error_comparison.png}
            \caption{Prediction Error with Constant Electrical Parameters}
            \label{fig:constant_params_error}
        \end{figure}
        As shown in Figure \ref{fig:constant_params_error}, the prediction error when SoC is low, which means the behavior that battery resistance 
        increases with lower SoC, is not captured well. This indicates the importance of considering time-varying parameters in battery modeling for 
        accurate predictions.

        \subsubsection{Numerical Accuracy}

        The calibration of electrical parameters of lithium batteries is crucial for accurate modeling and simulation. To assess the impact of numerical 
        accuracy on the calibration results, we add $5\%, 10\%, 20\%$ disturbance seperately to the electrical parameters $R_0, R_1, R_2$ and observe the
        resulting prediction errors in $1\text{A}$ constant current scenario. The results are shown below:
        \begin{figure}[H]
            \centering
            \begin{subfigure}{0.3\textwidth}
                \includegraphics[width=\textwidth]{sensitivity_error_R0.png}
            \end{subfigure}
            \hfill
            \begin{subfigure}{0.3\textwidth}
                \includegraphics[width=\textwidth]{sensitivity_error_R1.png}
            \end{subfigure}
            \hfill
            \begin{subfigure}{0.3\textwidth}
                \includegraphics[width=\textwidth]{sensitivity_error_R2.png}
            \end{subfigure}
            \caption{Sensitivity Analysis to Resistance}
        \end{figure}

        It can be observed that:
        \begin{itemize}
            \item All of the disturbances lead to prediction errors when SoC is low. Considering the fact that battery resistance increases with lower 
                  SoC, The prorpotional disturbance itself is scaled up, which leads to larger errors.
            \item The disturbance on $R_0$ leads to the most significant prediction error, and it is almost independent of time. $R_0$ is the basic 
                  internal resistance. It is not connected to any capacitors, so in constant current scanorio, it only affects output voltage.
            \item The disturbances on $R_1$ and $R_2$ have similar sharp pattern at the start. This is because they are connected to capacitors, which
                  cause transient response at the begining of discharge.
            \item The errors caused by $R_1$ are obviously more time-dependent than those caused by $R_2$. This is because $R_1$ is associated with $C_1$, 
                  which has a smaller time constant, leading to faster transient response. This indicates that there may be little accuracy loss if the 
                  model is simplified to 1-order, since the accuracy of $R_1$ is more critical for capturing the dynamics of the battery than $R_2$. 
        \end{itemize}

        The capacitance effect is not significant in constant current scenario. To further analyze the impact of numerical accuracy on $C_1, C_2$, we set
        a pulse current discharge scenario to simulate the battery behavior. The results are shown below:

