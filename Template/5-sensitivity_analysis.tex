\section{Model Evaluation and Further Discussions} 
    \subsection{Sensitivity Analysis}

        Smartphone batteries varies in capacity, voltage, and internal resistance. To understand how these variations affect the performance of the system,
        we conducted a sensitivity analysis by varying each parameter in equation \eqref{Thevenin_model_matrix_form} within realistic ranges.

        \subsubsection{Time-Varying Properties}
            Too much simplification leads to lose of actual behavior. For example, setting $R_0, R_1, R_2, C_1, C_2$ to constants rather than parameters 
            varies with $SOC$ can result in inaccurate predictions. Here is the error result if we simply set them as sample mean values:
            \begin{figure}[H]
                \centering
                \includegraphics[width=0.6\textwidth]{voltage_error_comparison.png}
                \caption{Prediction Error with Constant Electrical Parameters}
                \label{fig:constant_params_error}
            \end{figure}
            As shown in Figure \ref{fig:constant_params_error}, the prediction error when $SOC$ is low, which means the behavior that battery resistance 
            increases with lower $SOC$, is not captured well. This indicates the importance of considering time-varying parameters in battery modeling for 
            accurate predictions.

        \subsubsection{Numerical Accuracy}

            The calibration of electrical parameters of lithium batteries is crucial for accurate modeling and simulation. To assess the impact of 
            numerical accuracy on the calibration results, we add $5\%, 10\%, 20\%$ disturbance seperately to the electrical parameters $R_0, R_1, R_2$ 
            and observe the resulting prediction errors in $1\mathrm{A}$ constant current scenario. The results are shown below:
            \begin{figure}[H]
                \centering
                \begin{subfigure}{0.3\textwidth}
                    \includegraphics[width=\textwidth]{sensitivity_error_R0.png}
                \end{subfigure}
                \hfill
                \begin{subfigure}{0.3\textwidth}
                    \includegraphics[width=\textwidth]{sensitivity_error_R1.png}
                \end{subfigure}
                \hfill
                \begin{subfigure}{0.3\textwidth}
                    \includegraphics[width=\textwidth]{sensitivity_error_R2.png}
                \end{subfigure}
                \caption{Sensitivity Analysis to Resistance}
            \end{figure}

            It can be observed that:
            \begin{itemize}
                \item All of the disturbances lead to prediction errors when $SOC$ is low. Considering the fact that battery resistance increases with 
                      lower $SOC$, The prorpotional disturbance itself is scaled up, which leads to larger errors.
                \item The disturbance on $R_0$ leads to the most significant prediction error, and it is almost independent of time. $R_0$ is the basic 
                      internal resistance. It is not connected to any capacitors, so in constant current scanorio, it only affects output voltage.
                \item The disturbances on $R_1$ and $R_2$ have similar sharp pattern at the start. This is because they are connected to capacitors, which
                      cause transient response at the beginning of discharge.
                \item The errors caused by $R_1$ are obviously more time-dependent than those caused by $R_2$. This is because $R_1$ is associated with 
                      $C_1$, which has a smaller time constant, leading to faster transient response. However, this doesn't mean $R_2$ is less important, 
                      as it affects the short-term behavior of the battery during transient events discussing later.
            \end{itemize}

            The capacitance effect is not significant in constant current scenario. To further analyze the impact of numerical accuracy on $C_1, C_2$, we 
            set a pulse current discharge scenario ($1 \mathrm{A}, 10\mathrm{s}, SOC = 80\%$) to simulate the battery behavior. The results are shown 
            below:
            \begin{figure}[H]
                \centering
                \begin{subfigure}{0.45\textwidth}
                    \includegraphics[width=\textwidth]{pulse_voltage_error_C1.png}
                \end{subfigure}
                \hfill
                \begin{subfigure}{0.45\textwidth}
                    \includegraphics[width=\textwidth]{pulse_voltage_error_C2.png}
                \end{subfigure}
                \caption{Sensitivity Analysis to Capacitance}
            \end{figure}

            It can be observed that the effect of $C_1$ is faster than that of $C_2$, which is consistent with our previous assumptions.

            Let $\tau = RC$ is the time constant of the RC circuit that describes how faster an RC circuit responds. When $SOC = 80\%$, we have 
            $\tau_1 = 0.96\mathrm{s}$ and $\tau_2 = 8.84\mathrm{s}$. Therefore, during the 10s discharge pulse, $C_1$ will have a more immediate effect 
            on the voltage response compared to $C_2$, but over a longer period, $C_2$ will also significantly influence the voltage recovery after the 
            pulse ends. Therefore, both capacitances play important roles in capturing the transient behavior of the battery under pulse discharge 
            conditions, which supports our use of 2-order Thevenin model.

    \subsection{Strength and Weakness}
        \subsubsection{Strength}
            \begin{itemize}
                \item 2-order Thevenin model is more accurate than Rint model or 1-order Thevenin model in characterizing the battery performance. More 
                      electrochemical effects are considered and graded to different RC networks, which makes the model more reasonable.
                \item The model effectively captures the short-term relationship between smartphone power consumption, battery performance, and 
                      environmental factors such as temperature.
                \item The integration of empirical data from real-world usage scenarios enhances the model's accuracy and applicability. 
            \end{itemize}
        \subsubsection{Weakness}
            \begin{itemize}
                \item Long-term effects are not fully captured. The battery degradation model is relatively simplistic, relying on a linear approximation 
                      that may not fully capture the complex electrochemical processes involved in battery aging.
                \item Some of the assumptions are still idealized. For example, the model assumes a constant temperature during operation. However, 
                      smartphones often do not work in a equilibrium state. The temperature distribution around the smartphone may vary significantly, and
                      cause unignorable effects.
                \item The model may not generalize well to all smartphone models and usage patterns because of the limited availability of high-quality 
                      datasets.
            \end{itemize}
    \subsection{Further Work}
        \begin{itemize}
            \item \textbf{Consider the temperature distribution in the phone}. The current model assumes a uniform temperature for easy. Actually, the 
                   temperature distribution in the phone is not uniform. Processors are almost always hotter than other components, and the heat battery
                   itself generates may be not such important, especially in heavy usage scenarios like this figure shows. ($Bi$ is the Biot number, which
                   indicates an object's internal resistance over convective.)
                   \begin{figure}[H]
                    \centering
                    \includegraphics[width=0.6\textwidth]{temperature_model.png}
                    \caption{Comparison between Idealized and More Realistic Temperature Distribution}
                   \end{figure}
            \item \textbf{Consider charging scenarios}. The current model only focuses on the discharging process. However, charging is also an important
                   part of battery usage. Fast charging usually cause significant heat generation. Different charging methods (e.g., wireless charging) may
                   also have different effects on battery performance and degradation.
        \end{itemize}

