%美赛模板:正文部分

%模板中需要使用Xelatex编译两次,所以可以反复尝试编译
\documentclass[12pt]{article}  % 官方要求字号不小于 12 号,此处选择 12 号字体
% \linespread{1.1} %可选:设置行间距1.1倍
% 本模板不需要填写年份,以当前电脑时间自动生成
% 请在以下的方括号中填写队伍控制号
\usepackage[2618558]{easymcm}  % 载入 EasyMCM 模板文件
\problem{A}  % 请在此处填写题号
% \usepackage{mathptmx}  % 这是 Times 字体,中规中矩 
\usepackage{palatino}  % mathpazo 这palatino是 COMAP 官方杂志采用的更好看的 Palatino 字体,可替代以上的 mathptmx 宏包
\usepackage{pdfpages}
\usepackage{longtable} %长表格包
\usepackage{tabu}
\usepackage{threeparttable}
\usepackage{listings} %代码排版包
\usepackage{paralist} %压缩列表环境包(紧凑列表)
\usepackage{makecell} %表格换行包
\usepackage{multirow} %表格排版包
\usepackage[version=4]{mhchem}   %化学公式排版包
%\usepackage{fourier} %使用花里胡哨符号的宏包
%------------------------------------------代码排版包(待研究)
%    \usepackage{minted}  %代码排版包
%    \usepackage{caption} %浮动体包
%    \usepackage{xcolor}
\usepackage{matlab-prettifier} %MATLAB代码排版包
\usepackage{tcolorbox} %美丽盒子包
%    \tcbuselibrary{skins}
%    \tcbuselibrary{minted}
\tcbuselibrary{listings,skins,breakable,xparse}
%    \usemintedstyle{lovelace} %特定代码配色环境
%\tcbuselibrary{breakable}


\usepackage{pgfornament} %花样装饰包(待研究)
\usepackage[ruled,linesnumbered]{algorithm2e} %伪代码排版包
\usepackage{tikz} %绘图包
\usetikzlibrary{arrows.meta, positioning, calc}
\usepackage{wrapfig} %图文环绕包
\usepackage{float} %精确位置控制包
\usepackage{lettrine} %首字下沉包
%-------------------------------------------交叉引用实现包
\usepackage{hyperref}
\usepackage{cleveref}

\graphicspath{{img/}}          % 此处{img/}为相对路径,注意加上“/”
\let\itemize\compactitem{}
\let\enditemize\endcompactitem{}
%-------------------------------------------附录代码排版盒子(起)
 \tcbset{%
 mylist/.style = {%
  colframe = gray,
  colback = white,
  coltitle = red!50!yellow!3!white,
  colbacktitle = white,
  listing only,
  attach boxed title to top center = {yshift = -\tcboxedtitleheight/2},
  enhanced,
  drop fuzzy shadow, % shadow around listings
  left = 6.5mm, % distance between left rule and line number 
  breakable, % enable listing box to break by page
  %enhanced jigsaw, % box not being closed when broken by page
  fonttitle=\small\bfseries\color{black}, % customise font of listing title
  before skip=20pt plus 2pt, % vertical space between listings and text
  after skip=20pt plus 2pt,
 },
 example/.style 2 args = {%
  mylist,
  title = {Listing \thetcbcounter: #1},
  label = {#2},
 },
 }
 \newtcblisting[auto counter, number within = section, list inside = mcode]{matlab}[3][]{%
  listing options = {%
   style = Matlab-editor,
   numbers = left,
   numberstyle = \footnotesize\color{darkgray}\fontfamily{pcr}\selectfont,
   basicstyle = \footnotesize\fontfamily{pcr}\selectfont,
   breaklines = true,  % 启用自动换行
   breakatwhitespace = true,  % 在空格处换行
   postbreak = \mbox{\textcolor{gray}{$\hookrightarrow$}\space}  % 换行指示符
  },
  overlay = {\begin{tcbclipinterior}\fill[blue!15!white] (frame.south west) rectangle ([xshift=5.3mm]frame.north west);\end{tcbclipinterior}},
  example = {#2}{#3}, #1,
 }
 
 % Python代码盒子定义 - 与MATLAB配色统一
 \newtcblisting[auto counter, number within = section, list inside = pcode]{python}[3][]{%
  listing options = {%
   language = Python,
   numbers = left,
   numberstyle = \footnotesize\color{darkgray}\fontfamily{pcr}\selectfont,
   basicstyle = \footnotesize\fontfamily{pcr}\selectfont,
   keywordstyle = \color{darkblue}\bfseries,
   commentstyle = \color{darkgreen},
   stringstyle = \color{red!60},
   showstringspaces = false,
   emphstyle = \color{orange!60},
   emph = {self, cls, True, False, None},
   breaklines = true,  % 启用自动换行
   breakatwhitespace = true,  % 在空格处换行
   postbreak = \mbox{\textcolor{gray}{$\hookrightarrow$}\space}  % 换行指示符
  },
  overlay = {\begin{tcbclipinterior}\fill[blue!15!white] (frame.south west) rectangle ([xshift=5.3mm]frame.north west);\end{tcbclipinterior}},
  example = {#2}{#3}, #1,
 }
%-------------------------------------------附录代码排版盒子(止)

% AI报告专用代码高亮样式(无边框,支持跨页)
\lstdefinestyle{ai_python_style}{
    language = Python,
    numbers = none,  % 去掉行号
    basicstyle = \footnotesize\fontfamily{pcr}\selectfont,
    keywordstyle = \color{darkblue}\bfseries,
    commentstyle = \color{darkgreen},
    stringstyle = \color{red!60},
    showstringspaces = false,
    emphstyle = \color{orange!60},
    emph = {self, cls, True, False, None},
    breaklines = true,
    breakatwhitespace = true,
    postbreak = \mbox{\textcolor{gray}{$\hookrightarrow$}\space},
    frame = none,  % 无边框
    backgroundcolor = \color{white},
    xleftmargin = 0pt,
    xrightmargin = 0pt,
    framexleftmargin = 0pt,
    framexrightmargin = 0pt,
    framextopmargin = 0pt,
    framexbottommargin = 0pt
}

\lstdefinestyle{ai_matlab_style}{
    style = Matlab-editor,
    numbers = none,  % 去掉行号
    basicstyle = \footnotesize\fontfamily{pcr}\selectfont,
    breaklines = true,
    breakatwhitespace = true,
    postbreak = \mbox{\textcolor{gray}{$\hookrightarrow$}\space},
    frame = none,  % 无边框
    backgroundcolor = \color{white},
    xleftmargin = 0pt,
    xrightmargin = 0pt,
    framexleftmargin = 0pt,
    framexrightmargin = 0pt,
    framextopmargin = 0pt,
    framexbottommargin = 0pt
}

\lstnewenvironment{ai_python}[1][]
    {\lstset{style=ai_python_style, #1}}
    {}

\lstnewenvironment{ai_matlab}[1][]
    {\lstset{style=ai_matlab_style, #1}}
    {}

\input{new_command.tex} %载入新命令
%\newcommand{\upcite}[1]{\textsuperscript{\textsuperscript{\cite{#1}}}} %
\title{Better Battery Usage: \\ From the Insights of Modeling}  % 标题

% 如需要修改题头(默认为 MCM/ICM),请使用以下命令(此处修改为 MCM)
%\renewcommand{\contest}{MCM}


%--------------------------------------------文档开始
\begin{document}

% 此处填写摘要内容
\begin{abstract}

    It is widely complained that smartphone battery level is inaccurate and drains too fast during daily use. Actually, smartphone Li-ion batteries are 
    complex electrochemical systems influenced by various factors, such as temperature, power demand, and device heating. To optimize battery usage and
    extend battery life, it is crucial to understand how these factors affect battery performance. In this paper, we develop an equivalent model to 
    analyze and predict the performance of Li-ion batteries in smartphones under various operating conditions.

    We started by constructing a simplest reasonable model based on ideal battery assumptions, which provides a foundational understanding of battery 
    behavior. A 2-order Thevenin model is built without considering temperature, power demand, smartphone heating and other factors. Then dataset from 
    related HPPC tests are used to identify the model parameters. Because the differential equation could not be solved analytically, numerical methods are 
    employed to validate the model, showing logical and intuitive results.

    Next, we refined the model by incorporating real-world factors. The Arrhenius equation is introduced to account for temperature effects on battery 
    performance. It is observed that the internal resistance increases exponentially with decreasing temperature, leading to higher voltage drops under 
    load. We also considered the decomposition of smartphone power consumption, revealing that components like the data network and CPU significantly impact 
    battery drain. With these results, we further analyzed smartphones' self-heating effect during operation, which in turn affects battery temperature and
    performance. All of the results above should answer Problem 1 and Problem 2.

    Then, a sensitivity analysis was conducted to identify the most influential parameters affecting battery life and performance. It has not escaped our 
    eyes that $SOC$-dependent(so time-dependent) parameters, such as internal resistance and capacity, capture the sudden degradation behavior of 
    batteries at low $SOC$. Simply setting those parameters as constants may lead to significant errors in long-term predictions. We also applied 
    numerical distribution to parameters, quantifying the uncertainty in model outputs due to parameter variability in different scenarios. This part 
    should answer Problem 3.

    Finally, based on the insights gained from our modeling and analysis, we provided practical recommendations for optimizing battery usage in smartphones.
    This part should answer Problem 4.

    \vspace{5pt}  %mm	毫米	1 mm = 2.845 pt   pt 点	1 pt = 0.351 mm
    \textbf{Keywords: Smartphone; Li-ion Battery; 2-Order Thevenin Model; Arrhenius Equation; Sensitivity Analysis}

\end{abstract}
\maketitle  % 生成 Summary Sheet

\tableofcontents  % 生成目录
\section{Introduction}
    \subsection{Problem Background}
        Smartphone has become an indispensable part of our daily lives. What plays a key role in user experience and smartphones' design is the 
        development of smartphone battery. From where consumers stand, we don't want to be anxious as battery's draining behaviour is affected by a 
        tremendous amount of factors, like the processer load and the ambient temperature. Still, as manufacturers, we need to provide the most accurate
        prediction of the as a competitive edge, and most importantly, to adapt subsequent measures to increase battery's lifespan, which helps a lot in 
        environmental sustainability. All of these are calling for more-detailed physical models and optimized algorithm predictions.	
        \begin{figure}[H]
            \centering
            \includegraphics[width=0.4\textwidth]{animate.png}
        \end{figure}
    \subsection{Restatement of the Problem}
        Based on the background, our aim is to establish an algorithm from the very physical essence of battery theory and user's data, through which we
        could finally help users to predict the time-to-empty under different circumstances and adapt adequate optimizations. So we need to address:
        \begin{enumerate}
            \item \textbf{Continuous-Time Model}: Deriving a set of differential equations from lithium battery's principle, establishing an ideal and 
                                                  then an actual model that returns the $SOC$ as a function of time, and using collected data to validate 
                                                  and support them.
            \item \textbf{Time-to-Empty Predictions}: Approximating the time-to-empty from different initial charge level, and introduce complex 
                                                      conditions to find the differences.
            \item \textbf{Sensitivity Analysis}: Introducing fluctuations of usage patterns or making changes to some flexible parameters or the 
                                                 assumptions to examine the model for lithium batteries is solid.
            \item \textbf{Recommendations}: Finding out optimization strategies of improving battery life, considering battery aging's impact on effective
                                            capacity's reduction, and trying to generalize to other portable devices. 
        \end{enumerate}
    \subsection{Our Work}
        \begin{figure}[H]
            \centering
            \includegraphics[width=0.8\textwidth]{flowchart.png}
            \caption{Flowchart of Our Work}
        \end{figure}
\input{2-notations}
\section{Ideal Battery}
    \subsection{Assumptions}
        To consider an simplist ideal battery model, the following assumptions are made:
        \begin{itemize}
            \item The battery is kept at a constant environment, so the temperature effects are neglected.
            \item The battery has no energy loss during charging and discharging.
            \item The battery has infinite cycle life, so the degradation effects are neglected and all the parameters are constant.
        \end{itemize}

    \subsection{Thevenin Model}
        A Li-ion battery system can be extremely complex, involving electrochemical, thermal and mechanical processes. However, for system-level studies,
        an equivalent circuit model is often used to represent the battery behavior. The Thevenin model is a widely used equivalent circuit model that 
        captures the dynamic response of the battery voltage during charge and discharge cycles. The model is described as follows:

        % \begin{figure}[H]
        %     \centering
        %     \includegraphics[width=0.5\textwidth]{Thevenin_model.png}   % TODO: figure
        %     \caption{Thevenin Model of a Li-ion Battery}
        %     \label{fig:Thevenin_model}
        % \end{figure}

        According to experimental experiences, the SoC is positively correlated with the open-circuit voltage $U_{oc}$, so we can express SoC as a function
        of $U_{oc}$:
        \begin{equation}
            SoC = f(U_{oc})
        \end{equation}
        in which $f(\cdot)$ can be obtained through curve fitting based on experimental data.

        The relationship between $U_{oc}$ and other parameters in the Thevenin model can be expressed with Kirchhoff's laws:
        \begin{equation}
            E = U_{oc} + I R_{0} + U_{1}
        \end{equation}
        in which $E$ is the electromotive force.

        For $U_1$ we have:
        \begin{equation}
            I = C_{1} \frac{dU_{1}}{dt} + \frac{U_{1}}{R_{1}}
        \end{equation}

        Differentiateing the SoC's definition with respect to time, we get:
        \begin{equation}
            \frac{d(SoC)}{dt} = -\frac{I}{Q_{max}}
        \end{equation}
        where $Q_{max}$ is the maximum capacity of the battery.

        Combining the above equations, we can derive the complete Thevenin model:
        \begin{equation}
            \begin{cases}
                \displaystyle \frac{d(SoC)}{dt} = -\frac{I}{Q_{max}}                            \\
                \displaystyle \frac{dU_{1}}{dt} = -\frac{U_{1}}{R_{1} C_{1}} + \frac{I}{C_{1}}  \\
                E = U_{oc} + I R_{0} + U_{1}                                                    \\
                SoC = f(U_{oc}) 
            \end{cases}
        \end{equation}
        or in matrix form:
        \begin{equation}
            \frac{d}{dt}
            \begin{bmatrix}
                SoC  \\
                U_{1}
            \end{bmatrix}
            =
            \begin{bmatrix}
                0                      & 0                     \\
                0                      & \displaystyle -\frac{1}{R_{1} C_{1}}
            \end{bmatrix}
            \begin{bmatrix}
                SoC  \\
                U_{1}
            \end{bmatrix}
            +
            \begin{bmatrix}
                \displaystyle -\frac{1}{Q_{max}}    \\
                \displaystyle  \frac{1}{C_{1}}
            \end{bmatrix}
            I
        \end{equation}
    \subsection{Simulations}

    \subsection{Result}
\section{Actual Battery}
    \subsection{Assumptions}
        An actual battery is much more complex than the idealized model presented in the previous section. In this section, we will consider more factors
        that affect the performance and behavior of real batteries:
        \begin{itemize}
            \item \textbf{Temperature}: The performance of a battery can vary significantly with temperature. At low temperatures, the internal resistance
                                        increases, leading to reduced capacity and power output. Conversely, high temperatures can enhance performance but
                                        may also accelerate degradation.
            \item \textbf{Complex Power Profile}: Real batteries often experience varying power demands, which can affect their efficiency and lifespan.
                                                  High performance demands can lead to increased heat generation, which int turns affect battery 
                                                  temperature.
            \item \textbf{Shutdown Voltage}: Electronics need a minimum voltage to operate correctly. If the battery voltage drops(steadily or suddenly) 
                                             below this threshold, the device will shutdown even if SoC is not zero.
        \end{itemize}

    \subsection{Temperature Behavior of Battery}
        Actual battery behavior is significantly influenced by temperature variations. The most common parameter affected by temperature is the internal 
        resistance. Using the Arrhenius' equation, we can model the temperature dependence of internal resistance as follows:
        \begin{equation}
            R = R_0 \cdot e^{ - \dfrac{E_a}{R_u T}}
        \end{equation}
        where $R_0$ is a reference resistance, $E_a$ is the activation energy, $R_u$ is the universal gas constant and $T_0$ is a reference temperature.
        
        To conduct the regression, rewrite the equation in linear form by taking the natural logarithm:
        \begin{equation}
            \ln R  = - \frac{E_a}{R_u} \cdot \dfrac{1}{T} + \ln R_0
        \end{equation}

        The Samsung INR21700 30T 3Ah Li-ion Battery Dataset contains experimental data at various temperatures. Using this dataset again, we perform a 
        linear regression, yielding the following results:

        \begin{figure}[H]
            \centering
            \begin{subfigure}{0.3\textwidth}
                \includegraphics[width=\linewidth]{3d_surface_R0.png}
                \caption{$R_0$, $E_a = 17.47 kJ/mol$}
            \end{subfigure}
            \hfill
            \begin{subfigure}{0.3\textwidth}
                \includegraphics[width=\linewidth]{3d_surface_R1.png}
                \caption{$R_1$, $E_a = 37.24 kJ/mol$}
            \end{subfigure}
            \hfill
            \begin{subfigure}{0.3\textwidth}
                \includegraphics[width=\linewidth]{3d_surface_R2.png}
                \caption{$R_2$, $E_a = 15.09 kJ/mol$}
            \end{subfigure}
            \caption{3D Surface Fitting of Resistance with Temperature and SoC}
        \end{figure}

    \subsection{Battery in Different Work Loads}

    \subsection{Heat Transfer Model in High-Performance Scenarios}
        In this chapter, we discuss how the heat that battery generates affacts the temperature of battery itself, and in turn impacts its own 
        performance.

        %TODO: Assumptions about heat transfer model

        Let $Q$ be the internal heat power, $h$ be the convective heat transfer coefficient, $A$ be the surface area of the battery(assumed to be the area
        of the smartphone), $C$ be the heat capacity of the battery, $T$ be the battery temperature and $T_{env}$ be the environmental temperature(assumed
        constant). With Newton's law of cooling, we have:
        \begin{equation}
            \label{Law of cooling}
            Q = h A (T - T_{env}) + C \frac{dT}{dt}
        \end{equation}
        
        $Q$ can be expressed by the Joule's law as:
        \begin{equation}
            Q = I^2 R_{inter} = I^2 (R_0 + R_1 + R_2)
        \end{equation}

        Considering that $I$ and $R$ is time-dependent, equation \eqref{Law of cooling} is a first-order linear ODE. The solution is:
        \begin{equation}
            \label{Relation between T and t}
            T = T_{env} + \frac{1}{C} \int_0^t{e^{-\dfrac{2Ah(t - s)}{C}} Q(s) \mathrm{d}s}
        \end{equation}

        Substitute equation \eqref{Relation between T and t} into the Arrhenius equation, we can get the relation between internal resistance and time,
        and thus analyze how temperature affects battery performance over time in high-performance scenarios.
 
    \subsection{Answer to Question 1}

    \subsection{Answer to Question 2}
\section{Sensitivity Analysis}
    \subsection{Analysis of Basic Electrical Parameters}
        Smartphone batteries varies in capacity, voltage, and internal resistance. To understand how these variations affect the performance of the system,
        we conducted a sensitivity analysis by varying each parameter in equation \eqref{Thevenin_model_matrix_form} within realistic ranges.

        \subsubsection{Time-Varying Properties}
        Too much simplification leads to lose of actual behavior. For example, setting $R_0, R_1, R_2, C_1, C_2$ to constants rather than parameters 
        varies with SoC can result in inaccurate predictions. Here is the error result if we simply set them as sample mean values:
        \begin{figure}[H]
            \centering
            \includegraphics[width=0.6\textwidth]{voltage_error_comparison.png}
            \caption{Prediction Error with Constant Electrical Parameters}
            \label{fig:constant_params_error}
        \end{figure}
        As shown in Figure \ref{fig:constant_params_error}, the prediction error when SoC is low, which means the behavior that battery resistance 
        increases with lower SoC, is not captured well. This indicates the importance of considering time-varying parameters in battery modeling for 
        accurate predictions.

        \subsubsection{Numerical Accuracy}

        The calibration of electrical parameters of lithium batteries is crucial for accurate modeling and simulation. To assess the impact of numerical 
        accuracy on the calibration results, we add $5\%, 10\%, 20\%$ disturbance seperately to the electrical parameters $R_0, R_1, R_2$ and observe the
        resulting prediction errors in $1\text{A}$ constant current scenario. The results are shown below:
        \begin{figure}[H]
            \centering
            \begin{subfigure}{0.3\textwidth}
                \includegraphics[width=\textwidth]{sensitivity_error_R0.png}
            \end{subfigure}
            \hfill
            \begin{subfigure}{0.3\textwidth}
                \includegraphics[width=\textwidth]{sensitivity_error_R1.png}
            \end{subfigure}
            \hfill
            \begin{subfigure}{0.3\textwidth}
                \includegraphics[width=\textwidth]{sensitivity_error_R2.png}
            \end{subfigure}
            \caption{Sensitivity Analysis to Resistance}
        \end{figure}

        It can be observed that:
        \begin{itemize}
            \item All of the disturbances lead to prediction errors when SoC is low. Considering the fact that battery resistance increases with lower 
                  SoC, The prorpotional disturbance itself is scaled up, which leads to larger errors.
            \item The disturbance on $R_0$ leads to the most significant prediction error, and it is almost independent of time. $R_0$ is the basic 
                  internal resistance. It is not connected to any capacitors, so in constant current scanorio, it only affects output voltage.
            \item The disturbances on $R_1$ and $R_2$ have similar sharp pattern at the start. This is because they are connected to capacitors, which
                  cause transient response at the begining of discharge.
            \item The errors caused by $R_1$ are obviously more time-dependent than those caused by $R_2$. This is because $R_1$ is associated with $C_1$, 
                  which has a smaller time constant, leading to faster transient response. This indicates that there may be little accuracy loss if the 
                  model is simplified to 1-order, since the accuracy of $R_1$ is more critical for capturing the dynamics of the battery than $R_2$. 
        \end{itemize}

        The capacitance effect is not significant in constant current scenario. To further analyze the impact of numerical accuracy on $C_1, C_2$, we set
        a pulse current discharge scenario to simulate the battery behavior. The results are shown below:


\section{User Recommendations}
\label{sec:recommendations}

Based on the power consumption model and battery degradation analysis, we provide practical recommendations for smartphone users. These recommendations are organized into two categories: \textbf{short-term advice} for extending single-charge runtime, and \textbf{long-term advice} for maximizing battery lifespan.

\subsection{Short-Term Recommendations: Extending Battery Runtime}

For users concerned with maximizing daily battery life, we quantify the effect of various power-saving measures based on our power consumption model. Assuming a constant power draw $P$, the simplified runtime can be expressed as:
\begin{equation}
T_{\text{runtime}} = \frac{E_{\text{capacity}}}{P_{\text{total}}}
\label{eq:runtime}
\end{equation}
where $E_{\text{capacity}} = 15.2~\text{Wh}$ for our reference 4000~mAh battery.

Using the power model coefficients from Section~\ref{sec:power_model}, we calculate the battery life improvement for each power-saving action relative to a web browsing baseline (1.08~W). The results are shown in Figure~\ref{fig:power_saving}.

\begin{figure}[H]
    \centering
    \includegraphics[width=0.78\textwidth]{img/power_saving_effects.pdf}
    \caption{Effect of power-saving measures on battery runtime}
    \label{fig:power_saving}
\end{figure}

Based on these results, we briefly summarize the power-saving actions:

\begin{itemize}
    \item \textbf{WiFi vs. Cellular}: Switching from cellular to WiFi reduces power by 0.696~W, improving battery life by 181\%.
    \item \textbf{Screen Off}: Turning off the display when not in use saves approximately 0.554~W (at 50\% brightness), extending runtime by 105\%.
    \item \textbf{Lower Brightness}: Reducing screen brightness from 100\% to 30\% saves 0.412~W, a 62\% improvement.
    \item \textbf{Disable Audio}: Muting audio playback reduces power by 0.397~W, adding 53\% to battery life.
    \item \textbf{Power-Saving Mode}: While providing only +7\% direct improvement, this mode is often the simplest solution as it automatically applies multiple optimizations above---reducing brightness, limiting background activity, and restricting network usage.
\end{itemize}

\subsection{Long-Term Recommendations: Maximizing Battery Lifespan}

Battery capacity degrades over charge-discharge cycles. We adopt a simplified linear aging model:
\begin{equation}
Q(n) = Q_0 \cdot (1 - \alpha \cdot n)
\label{eq:aging}
\end{equation}
where $n$ is the number of full equivalent cycles and $\alpha$ is the capacity fade rate per cycle.

Using the NASA Battery Dataset, we fitted $\alpha = 0.000411$ per cycle ($R^2 = 0.97$), indicating 0.041\% capacity loss per cycle. At 500 cycles, capacity retention is approximately 79.5\%, consistent with the industry standard of 80\% end-of-life threshold.

Figure~\ref{fig:lifespan} shows the time required for different user types to reach various capacity thresholds, based on their daily energy consumption patterns.

\begin{figure}[H]
    \centering
    \includegraphics[width=0.80\textwidth]{img/battery_lifespan_users.pdf}
    \caption{Time to reach different capacity thresholds}
    \label{fig:lifespan}
\end{figure}

Light users (6~Wh/day) can maintain 80\% capacity for 3.4 years, comfortably exceeding typical 2-year warranty periods. Normal users (10~Wh/day) reach the 80\% threshold at approximately 2.0 years---right at the warranty boundary. Heavy users (18~Wh/day) may need battery replacement within 1.1 years to maintain 80\% capacity.

To maximize long-term battery health, we recommend:
\begin{itemize}
    \item \textbf{Reduce daily energy consumption}: Following the short-term recommendations above reduces cycle frequency and extends lifespan proportionally.
    \item \textbf{Maintain 20\%--80\% charge range}: Avoiding extreme charge states reduces electrochemical stress on the battery.
    \item \textbf{Avoid high-temperature charging}: Elevated temperatures accelerate degradation according to Arrhenius kinetics; remove phone cases during charging.
    \item \textbf{Use standard-speed charging}: Fast charging generates more heat; reserve it for emergencies.
\end{itemize}

By combining short-term power management with long-term health practices, users can maintain battery capacity above 80\% for 2--3 years under normal usage conditions.


\newpage
\section*{Notation}
\addcontentsline{toc}{section}{Notation}

The following tables summarize all symbols used in this paper.

\renewcommand{\arraystretch}{1.2}

\begin{table}[H]
    \centering
    \caption{Battery, Thermal, and Power Parameters}
    \resizebox{\textwidth}{!}{
    \begin{tabular}{l l l l}
        \hline
        \textbf{Parameter} & \textbf{Symbol} & \textbf{Physical Meaning} & \textbf{Unit/Range} \\
        \hline
        \multicolumn{4}{l}{\textit{Battery Electrical Parameters}} \\
        \hline
        State of Charge          & $SOC$     & Battery capacity percentage remaining         & $0 \sim 1$     \\
        Open Circuit Voltage     & $U_{OC}$  & Voltage when no current flows                 & V              \\
        Terminal Voltage         & $U_t$     & Measured voltage at battery terminals         & V              \\
        Discharge Current        & $I$       & Current flowing out during discharge          & A              \\
        Maximum Capacity         & $Q_{max}$ & Total charge capacity of battery              & Ah             \\
        Ohmic Resistance         & $R_0$     & Ionic and electronic conduction resistance    & $\Omega$       \\
        Polarization Resistance 1 & $R_1$    & Electrochemical polarization (fast)           & $\Omega$       \\
        Polarization Resistance 2 & $R_2$    & Concentration polarization (slow)             & $\Omega$       \\
        Polarization Capacitance 1 & $C_1$   & Capacitance for fast dynamics                 & F              \\
        Polarization Capacitance 2 & $C_2$   & Capacitance for slow dynamics                 & F              \\
        Polarization Voltage 1   & $U_1$     & Voltage across $R_1$-$C_1$ branch             & V              \\
        Polarization Voltage 2   & $U_2$     & Voltage across $R_2$-$C_2$ branch             & V              \\
        Time Constant 1          & $\tau_1$  & $R_1 C_1$, fast response time                 & s              \\
        Time Constant 2          & $\tau_2$  & $R_2 C_2$, slow response time                 & s              \\
        \hline
        \multicolumn{4}{l}{\textit{Thermal Parameters}} \\
        \hline
        Activation Energy        & $E_a$           & Energy barrier in Arrhenius equation          & kJ/mol             \\
        Universal Gas Constant   & $R_u$           & Physical constant (8.314)                     & J/(mol$\cdot$K)    \\
        Battery Temperature      & $T$             & Real-time battery temperature                 & K                  \\
        Environment Temperature  & $T_{env}$       & Ambient temperature (constant)                & K                  \\
        Maximum Temperature      & $T_{max}$       & Peak temperature during operation             & K                  \\
        Internal Heat Power      & $Q$             & Total heat generation rate                    & W                  \\
        Battery Joule Heat       & $Q_{battery}$   & Heat from internal resistance                 & W                  \\
        Processor Heat           & $Q_{processor}$ & Heat from CPU and circuits                    & W                  \\
        Other Heat               & $Q_{other}$     & Heat from other modules                       & W                  \\
        Heat Transfer Coefficient & $h$            & Convective heat transfer coefficient          & W/(m$^2\cdot$K)    \\
        Surface Area             & $A$             & Heat dissipation area of device               & m$^2$              \\
        Heat Capacity            & $C$             & Thermal mass of battery system                & J/K                \\
        Heat Efficiency          & $\eta$          & Fraction of electrical power to heat          & --                 \\
        Biot Number              & $Bi$            & Dimensionless heat transfer parameter         & --                 \\ 
        \hline
        \multicolumn{4}{l}{\textit{Power Components}} \\
        \hline
        Total Power              & $P_{total}$   & Sum of all power components                   & W              \\
        Screen Power             & $P_{screen}$  & Display system power consumption              & W              \\
        CPU Power                & $P_{CPU}$     & Processor power consumption                   & W              \\
        Network Power            & $P_{network}$ & Wireless communication power                  & W              \\
        GPS Power                & $P_{GPS}$     & GPS receiver power consumption                & W              \\
        Audio Power              & $P_{audio}$   & Audio playback power consumption              & W              \\
        Mode Adjustment          & $P_{mode}$    & Power adjustment from system modes            & W              \\
        \hline
    \end{tabular}
    }
\end{table}

\begin{table}[H]
    \centering
    \caption{State Variables and Model Coefficients}
    \resizebox{\textwidth}{!}{
    \begin{tabular}{l l l l}
        \hline
        \textbf{Parameter} & \textbf{Symbol} & \textbf{Physical Meaning} & \textbf{Unit/Range} \\
        \hline
        \multicolumn{4}{l}{\textit{State Variables}} \\
        \hline
        Screen State             & $S$           & Display on/off indicator                      & $\{0, 1\}$     \\
        Brightness Level         & $B$           & Display brightness value                      & $0 \sim 255$   \\
        CPU Utilization          & $U$           & Fraction of CPU active time                   & $0 \sim 1$     \\
        Big Core Frequency       & $f_{big}$     & High-performance core frequency               & GHz            \\
        Small Core Frequency     & $f_{small}$   & Efficiency core frequency                     & GHz            \\
        Network Type             & $M$           & 0: WiFi, 1: Cellular                          & $\{0, 1\}$     \\
        GPS Activity             & $G$           & GPS on/off indicator                          & $\{0, 1\}$     \\
        Audio Playback           & $A$           & Audio active indicator                        & $\{0, 1\}$     \\
        Power Saving Mode        & $E$           & Power saving mode on/off                      & $\{0, 1\}$     \\
        Flight Mode              & $F$           & Flight mode on/off                            & $\{0, 1\}$     \\
        \hline
        \multicolumn{4}{l}{\textit{Model Coefficients ($\alpha_S, \alpha_B, ... > 0$; $\alpha_E, \alpha_F < 0$)}} \\
        \hline
        Screen Base Coefficient  & $\alpha_S$      & Base power when screen on                   & W              \\
        Brightness Coefficient   & $\alpha_B$      & Power per brightness unit                   & W              \\
        CPU Utilization Coefficient & $\alpha_U$   & Power per unit utilization                  & W              \\
        Big Core Coefficient     & $\alpha_{big}$  & Big core power at max frequency             & W              \\
        Small Core Coefficient   & $\alpha_{small}$& Small core power at max frequency           & W              \\
        Network Coefficient      & $\alpha_M$      & Cellular power vs WiFi baseline             & W              \\
        GPS Coefficient          & $\alpha_G$      & GPS active power                            & W              \\
        Audio Coefficient        & $\alpha_A$      & Audio playback power                        & W              \\
        Power Saving Coefficient & $\alpha_E$      & Power reduction (negative)                  & W              \\
        Flight Mode Coefficient  & $\alpha_F$      & Power reduction (negative)                  & W              \\
        \hline
        \multicolumn{4}{l}{\textit{Battery Aging}} \\
        \hline
        Current Capacity         & $Q(n)$          & Capacity after $n$ cycles                   & Ah             \\
        Initial Capacity         & $Q_0$           & Original battery capacity                   & Ah             \\
        Cycle Number             & $n$             & Number of full equivalent cycles            & --             \\
        Capacity Fade Rate       & $\alpha$        & Capacity loss per cycle                     & cycle$^{-1}$   \\
        \hline
        \multicolumn{4}{l}{\textit{Time and Evaluation Metrics}} \\
        \hline
        Time                     & $t$                 & Elapsed time from start                 & s              \\
        Battery Runtime          & $T_{\text{runtime}}$& Time until battery depletion            & h              \\
        Energy Capacity          & $E_{capacity}$      & Total energy capacity                   & Wh             \\
        Coefficient of Determination & $R^2$           & Variance explained by model             & $0 \sim 1$     \\
        Mean Absolute Error      & MAE                 & Average absolute prediction error       & same as $y$    \\
        Root Mean Square Error   & RMSE                & Root of average squared error           & same as $y$    \\
        \hline
    \end{tabular}
    }
\end{table}

\noindent
\textbf{Notes:} MAE $= \dfrac{1}{n}\sum_{i=1}^{n}|y_i - \hat{y}_i|$, RMSE $= \sqrt{\dfrac{1}{n}\sum_{i=1}^{n}(y_i - \hat{y}_i)^2}$, where $y_i$ and 
$\hat{y}_i$ denote actual and predicted values.


\newpage
\begin{thebibliography}{99}

% [1] Samsung 电池数据集 - 用于 SOC 模型构建和温度修正
\bibitem{samsung_dataset}
    Kollmeyer, P. (2020).
    Samsung INR21700 30T 3Ah Li-ion Battery Data [Data set].
    Mendeley Data.
    \url{https://doi.org/10.17632/9xyvy2njj3.2}

% [2] ECM 等效电路模型
\bibitem{ecm_model}
    Ouyang, Q., Chen, J., \& Zheng, J. (2023).
    State of Charge Estimation for Lithium-Ion Battery Based on the Fractional Order Equivalent Circuit Model.
    \textit{IEEE Access}, 11, 107175--107184.
    \url{https://ieeexplore.ieee.org/document/10287336}

% [3] 安卓设备多维使用记录数据集 - 用于功耗分析
\bibitem{android_dataset}
    Ferreira, D., Gonçalves, J., Kostakos, V., \& Dey, A. K. (2024).
    Android Phone Usage Dataset: Multi-day, Multi-device Usage Logs [Data set].
    Mendeley Data, V6.
    \url{https://doi.org/10.17632/bpsrw76hgx.6}

% [4] NASA 电池老化数据集 - 用于老化模型
\bibitem{nasa_dataset}
    Saha, B., \& Goebel, K. (2007).
    Li-ion Battery Aging Datasets.
    NASA Open Data Portal.
    \url{https://data.nasa.gov/dataset/li-ion-battery-aging-datasets}

% [5] 电池老化机制综述
\bibitem{battery_aging}
    Barré, A., Deguilhem, B., Grolleau, S., Gérard, M., Suard, F., \& Riu, D. (2013).
    A review on lithium-ion battery ageing mechanisms and estimations for automotive applications.
    \textit{Journal of Power Sources}, 241, 680--689.
    \url{https://doi.org/10.1016/j.jpowsour.2013.05.040}

\end{thebibliography}



\label{LastMainPage}  % 标记正文最后一页,用于页眉“Page X of Y”
\clearpage
\fancypagestyle{aipage}{%
  \fancyhf{}
  \lhead{\small \team}
  \chead{}
  \rhead{}
  \lfoot{}
  \cfoot{}
  \rfoot{}
  \renewcommand{\headrulewidth}{0.4pt}
}
\pagestyle{aipage}  % 只保留左边页眉,去掉右边页码
\section*{Report on Use of AI}
\addcontentsline{toc}{section}{Report on Use of AI}

This section documents authentic interactions between the team and AI assistant (GitHub Copilot) during the development of this paper. These examples demonstrate responsible AI usage focused on \textbf{data analysis}, \textbf{result interpretation}, and \textbf{document formatting}, while all core modeling work was completed independently by the team.

\subsection*{Battery Dataset Selection}

\begin{tcolorbox}[title=\textbf{Question: Using 18650 Data for Smartphone Modeling}, colback=blue!2!white, colframe=blue!30!black, fonttitle=\bfseries, breakable]
We found some 18650 cylindrical cell datasets with complete HPPC test data. Can we use 18650 data directly for smartphone battery modeling?
\end{tcolorbox}

\begin{tcolorbox}[title=\textbf{AI Response}, colback=green!2!white, colframe=green!30!black, fonttitle=\bfseries, breakable]
Using 18650 data directly for smartphone modeling is not ideal due to several differences:

\begin{itemize}
    \item \textbf{Form Factor:} 18650 is cylindrical with rigid metal casing, while smartphone batteries are flat pouch cells
    \item \textbf{Internal Resistance:} 18650 cells typically have higher resistance due to smaller electrode surface area
    \item \textbf{Thermal Characteristics:} Metal-cased 18650 cells have better heat dissipation; pouch cells are more thermally sensitive
    \item \textbf{Chemistry:} 18650 often uses NCA/NMC, while smartphones typically use LiCoO$_2$
\end{itemize}

These differences would introduce systematic errors. I recommend searching for smartphone-specific battery test data if available.
\end{tcolorbox}

\noindent\textbf{Team's Contribution:} Following this advice, we searched for smartphone-specific datasets and found the Samsung INR21700 30T battery dataset, which includes complete HPPC characterization at multiple temperatures ($0^\circ \mathrm{C}, 23^\circ \mathrm{C}, 45^\circ \mathrm{C}$). We selected this dataset for our parameter identification.

\subsection*{Dataset Overview and Statistics}

\begin{tcolorbox}[title=\textbf{Question: Analyzing Dataset Structure}, colback=blue!2!white, colframe=blue!30!black, fonttitle=\bfseries, breakable]
I have a large smartphone usage dataset. Can you help me analyze the overall data distribution and summarize what features are available?
\end{tcolorbox}

\begin{tcolorbox}[title=\textbf{AI Response}, colback=green!2!white, colframe=green!30!black, fonttitle=\bfseries, breakable]
Based on your dataset structure, here's an overview:

\begin{itemize}
    \item \textbf{Total records:} 3.24 million discharge data points
    \item \textbf{Target device:} Device ID \texttt{ab47b5f5ecded6b8} (largest data volume)
    \item \textbf{After filtering:} 2.96 million valid points (excluding battery $>$ 95\%)
    \item \textbf{Sampling result:} 591,637 observations for regression
\end{itemize}

Key features available: screen state and brightness, CPU utilization and core frequencies, network connection status, GPS/Bluetooth/audio states, and power-saving mode indicators.
\end{tcolorbox}

\noindent\textbf{Team's Contribution:} We designed the complete data processing pipeline, defined filtering criteria based on domain knowledge (excluding high battery levels to avoid trickle charging artifacts), and implemented the feature engineering logic.

\subsection*{Coefficient Reasonableness Analysis}

\begin{tcolorbox}[title=\textbf{Question: Validating Model Coefficients}, colback=blue!2!white, colframe=blue!30!black, fonttitle=\bfseries, breakable]
My regression model produced these coefficients: Screen On (0.250W), Big Core (1.125W), Small Core (0.650W), Mobile Network (0.696W), GPS (0.040W), Power Saving (-0.068W). Can you help analyze whether they are physically reasonable?
\end{tcolorbox}

\begin{tcolorbox}[title=\textbf{AI Response}, colback=green!2!white, colframe=green!30!black, fonttitle=\bfseries, breakable]
These coefficients appear physically reasonable:

\begin{enumerate}
    \item \textbf{Big Core $>$ Small Core:} Consistent with big.LITTLE architecture where performance cores consume more power than efficiency cores.
    \item \textbf{Mobile Network is significant:} Cellular RF transmission requires higher power than WiFi due to longer communication distances.
    \item \textbf{GPS is small:} GPS is primarily a receiver with low power consumption.
    \item \textbf{Power-saving modes are negative:} The small magnitudes suggest most savings come from reducing other parameters already captured in the model.
\end{enumerate}
\end{tcolorbox}

\noindent\textbf{Team's Contribution:} We derived the model structure from first principles (power superposition, CMOS dynamic power formula, DVFS scaling law). AI assisted in reviewing whether the estimated coefficients align with expected physical behavior.

\subsection*{Data Sampling Strategy}

\begin{tcolorbox}[title=\textbf{Question: Efficient Sampling for Large Datasets}, colback=blue!2!white, colframe=blue!30!black, fonttitle=\bfseries, breakable]
I have 3 million data points but need to reduce computational cost. What sampling approach would preserve data quality?
\end{tcolorbox}

\begin{tcolorbox}[title=\textbf{AI Response}, colback=green!2!white, colframe=green!30!black, fonttitle=\bfseries, breakable]
For time-series data, use systematic sampling (jump sampling) rather than random sampling:

\begin{ai_python}
df_sorted = df.sort_values('timestamp')
sample = df_sorted.iloc[::5]  # Every 5th point
\end{ai_python}

This preserves temporal distribution and maintains representation of different usage patterns.
\end{tcolorbox}

\noindent\textbf{Team's Contribution:} We determined the 1:5 sampling ratio based on computational constraints and validated that model coefficients remain stable across different sampling ratios.

\subsection*{LaTeX Equation Formatting}

\begin{tcolorbox}[title=\textbf{Question: Breaking Long Equations}, colback=blue!2!white, colframe=blue!30!black, fonttitle=\bfseries, breakable]
My power model equation is too long and overflows the page margin. How do I split it across lines?
\end{tcolorbox}

\begin{tcolorbox}[title=\textbf{AI Response}, colback=green!2!white, colframe=green!30!black, fonttitle=\bfseries, breakable]
Use the \texttt{aligned} environment inside \texttt{equation}:

\begin{verbatim}
\boxed{\begin{aligned}
    P_{total} &= 0.250S + 0.615S\frac{B}{255} + ... \\
    &\quad + 0.696M + 0.040G + ...
\end{aligned}}
\end{verbatim}
\end{tcolorbox}

\noindent\textbf{Team's Contribution:} The model equation and all physical terms were derived by the team. AI assisted only with LaTeX syntax.

\subsection*{LaTeX Table Column Width Adjustment}

\begin{tcolorbox}[title=\textbf{Question: Uniform Table Column Widths}, colback=blue!2!white, colframe=blue!30!black, fonttitle=\bfseries, breakable]
My notation table has inconsistent column widths, and the table is too narrow compared to the page width. How can I make all columns have uniform width and expand the table to fill the text width?
\end{tcolorbox}

\begin{tcolorbox}[title=\textbf{AI Response}, colback=green!2!white, colframe=green!30!black, fonttitle=\bfseries, breakable]
Use the \texttt{tabularx} package with equal-width columns:

\begin{verbatim}
\usepackage{tabularx}
\newcolumntype{Y}{>{\centering\arraybackslash}X}

\begin{tabularx}{\textwidth}{|Y|Y|Y|}
\hline
\textbf{Symbol} & \textbf{Description} & \textbf{Unit} \\
\hline
$R_0$ & Ohmic resistance & $\Omega$ \\
...
\end{tabularx}
\end{verbatim}

The \texttt{X} column type automatically distributes remaining width equally. Using \texttt{\textbackslash textwidth} ensures the table spans the full page width.
\end{tcolorbox}

\noindent\textbf{Team's Contribution:} We created all table content and symbol definitions. AI assisted with LaTeX table formatting syntax.

\subsection*{Summary of AI Usage}

\begin{table}[H]
\centering
\begin{tabular}{@{}lll@{}}
\toprule
\textbf{Category} & \textbf{AI Assistance} & \textbf{Team Responsibility} \\
\midrule
Core Modeling & --- & Thevenin model, Arrhenius, DVFS \\
Data Selection & Advised against 18650 & Found Samsung dataset \\
Data Analysis & Dataset statistics & Filtering, feature engineering \\
Result Interpretation & Coefficient review & Model fitting, validation \\
Documentation & LaTeX table/equation formatting & All technical writing \\
\bottomrule
\end{tabular}
\caption{Division of work between AI assistance and team contribution}
\end{table}

\subsection*{Declaration}

All theoretical frameworks in this paper---including the second-order RC equivalent circuit model, Arrhenius temperature dependence, CMOS dynamic power formula ($P \propto f^{2.5}$), Friis transmission equation for RF power, and Newton's law of cooling for thermal modeling---were \textbf{independently researched, derived, and implemented by the team members}. AI tools were used for comparing battery types, understanding dataset structure, reviewing coefficient reasonableness, and LaTeX formatting---reflecting responsible AI assistance for productivity enhancement without replacing human judgment in scientific analysis.


\end{document}